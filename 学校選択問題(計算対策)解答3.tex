\documentclass{jarticle}
\usepackage{amsmath}
\usepackage{ascmac}
\usepackage{cases}
\usepackage{amsmath}
\begin{document}
\begin{center}
第3回(学校選択問題の計算対策)回答
\end{center}
(1)
$
\begin{cases}
x^{2}+y^{2}=1\\
y+x=1
\end{cases}
$
\\
\\
\\
下の$y+x=1$を$y=-x+1$として、上の式に代入する.\\
\[x^{2}+y^{2}=1\]
\\
\[x^{2}+(-x+1)^{2}=1\]
\\
\[2x^{2}-2x=0\]
\\
\[2x(x-1)=0\]
\\
\[x=0,1\]
\\
これを$y+x=1$に代入して$y$の値を求める.
\\
\[(x,y)=(0,1),(1,0)\]
\\
\\
(2)
$6x+y=x+2y=1$は、
\[
\begin{cases}
6x+y=1\\
x+2y=1
\end{cases}
\]
と考えることができる.
\\
\\
この連立方程式を解くと$(x,y)=(\frac{1}{11},\frac{5}{11})$とわかる.
\\
\\
(3)
$x+2y=1,3x+4y=6,5x+ay=4$が一点で交わるということは、
\\
\\
$x+2y=1,3x+4y=6$の交点を$5x+ay=4$が通ることになる.
\\
\\
$x+2y=1,3x+4y=6$の交点は、連立すると求めることができ、$(x,y)=(4,-\frac{3}{2})$とわかる.
\\
\\
この点を$5x+ay=4$に代入すると$a=\frac{28}{3}$とわかる.


\end{document}