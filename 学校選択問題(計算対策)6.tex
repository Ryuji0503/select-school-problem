\documentclass{jsarticle}
\begin{document}
\begin{center}
学校選択問題(計算対策)第6回
\end{center}

(1)
\\
\\
$y=ax^{2}$とする.$ 0 \leq x \leq 2$,$-4\leq y \leq 0$のとき,aの値を求めよ
\\
\\
\hspace{10pt}(2)
\\
\\
$x$の値が$a$から$a+2$まで増加するとき,2つの関数$y=-4x+2$とy=$\frac{1}{3}x^{2}$の変化の割合が等しくなるような$a$の値を求めよ
\\
\\
\hspace{10pt}(3)(発展)
\\
\\
$y=x^{2}$とする.$a\leq x \leq 3$のとき,$0 \leq y \leq 9 $とする.$b$の値と$a$の取りうる値を求めよ

\end{document}